\documentclass[12pt]{article}

\usepackage{graphicx}
\usepackage{amssymb}
\usepackage{amsmath}
\usepackage[margin=1in]{geometry}
\usepackage{fancyhdr}
\usepackage{enumerate}
\usepackage[shortlabels]{enumitem}

\pagestyle{fancy}
\fancyhead[l]{Li Yifeng}
\fancyhead[c]{Homework \#4}
\fancyhead[r]{\today}
\fancyfoot[c]{\thepage}
\renewcommand{\headrulewidth}{0.2pt}
\setlength{\headheight}{15pt}

\newcommand{\bP}{\mathbb{P}}

\begin{document}
	
	\section*{Question 3}
	
	\noindent You have two unfair coins, one with the probability of heads equal to $p_1$ and the other with the probability of heads equal to $p_2$, where $p_2 \neq p_1$. In strategy A, you choose one coin at random and toss it twice. In strategy B, you toss both coins. What is the best strategy to maximize the probability of the event $E = \text{"the two tosses are both heads"}$?
	
	\begin{enumerate}[label={},leftmargin=0in]\item
		
		\subsection*{Solution}
			
			Let $\it{head}$ and $\it{tail}$ denote the results of a drop. Define the probability space as
		
			\[
			\begin{aligned}
				\Omega &= \{\mathrm{head},\,\mathrm{tail}\}\\
				\mathcal{F} &= \mathcal{P}(\Omega)\\
				\bP &:\enspace \text{the probability measure on $\mathcal{F}$}
			\end{aligned}
			\]
			
			Let $C_1$ and $C_2$ denote the events of selecting the first mentioned coin and the second mentioned coin, respectively, then we have
			
			\[
			\begin{aligned}
				\bP_{C_1}(\{\mathrm{head}\}) &= p_1\\
				\bP_{C_2}(\{\mathrm{head}\}) &= p_2
			\end{aligned}
			\]
			
			Let the first drop and the second drop be denoted by subscripts $1$ and $2$, respectively, then we have
			
			\[\bP_{C_{i_j}}(\{\mathrm{head}\}_k) = p_i,\quad i,j,k\in\{1,2\}\]
			
			We know that $C_{1_1}$, $C_{2_1}$, $C_{1_2}$ and $C_{2_2}$ are stochastically independent
			
			\[\bP(C_{1_1}\cap C_{2_1}\cap C_{1_2}\cap C_{2_2}) = \left(\frac{1}{2}\right)^4 = \bP(C_{1_1})\bP(C_{2_1})\bP(C_{1_2})\bP(C_{2_2})\]
			
			And we know that $\{C_{1_1},C_{2_1}\}$ and $\{C_{1_2},C_{2_2}\}$ are two partitions of $\Omega$, then we know that $\{head\}_1$ and $\{head\}_2$ are stochastically independent
			
			\[
			\begin{aligned}
				&\bP(\{head\}_1\cap \{head\}_2)\\
				= &\left(\sum_{j=1}^2\bP_{C_{j_1}}(\{\mathrm{head}\}_1)\bP(C_{j_1})\right)\left(\sum_{j=1}^2\bP_{C_{j_2}}(\{\mathrm{head}\}_2)\bP(C_{j_2})\right)\\
				= &\bP(\{\mathrm{head}\}_1)\bP(\{\mathrm{head}\}_2)
			\end{aligned}
			\]
			
			\subsubsection*{Strategy A: choose one coin at random and toss it twice}
			
			By applying the principle of symmetry, we have
			
			\[
			\begin{aligned}
				&\bP(C_{1_1}) = \bP(C_{2_1}) = \bP(C_{1_2}) = \bP(C_{2_2})\\
				= &\bP_{C_{1_1}}(C_{1_2}) = \bP_{C_{2_1}}(C_{1_2}) = \bP_{C_{1_2}}(C_{2_2}) = \bP_{C_{2_2}}(C_{2_2})\\
				= &\frac{1}{2}
			\end{aligned}
			\]
			
			We know that $\{C_{1_1},C_{2_1}\}$ and $\{C_{1_2},C_{2_2}\}$ are two partitions of $\Omega$, then we have the probability of event $E$ in strategy A
			
			\[
			\begin{aligned}
				\bP(E_A) &= \bP(\{\mathrm{head}\}_1\cap \{\mathrm{head}\}_2)\\
				&= \bP(\{\mathrm{head}\}_1)\bP(\{\mathrm{head}\}_2)\\
				&= \sum_{j=1}^2\left(\bP(C_{j_1})\bP_{C_{j_1}}(\{\mathrm{head}\}_1)\bP_{C_{j_2}}(\{\mathrm{head}\}_2)\right)\\
				&= \frac{{p_1}^2 + {p_2}^2}{2}
			\end{aligned}
			\]
			
			\subsubsection*{Strategy B: toss both coins}
			
			Because the order does not matter here, one can assume to drop the coin $C_1$ first with the coin $C_2$ second, then we have the probability of event $E$ in strategy B
			
			\[
			\begin{aligned}
				\bP(E_B) &= \bP(\{\mathrm{head}\}_1\cap \{\mathrm{head}\}_2)\\
				&= \bP(\{\mathrm{head}\}_1)\bP(\{\mathrm{head}\}_2)\\
				&= \bP_{C_{1_1}}(\{\mathrm{head}\}_1)\bP_{C_{2_2}}(\{\mathrm{head}\}_2)\\
				&= p_1p_2
			\end{aligned}
			\]
			
			\subsubsection*{Conclusion}
			
			By applying AM–GM inequality, we know that
			
			
			
		\subsection*{Answer}
		
			\[\boxed{}\]
	
	\end{enumerate}
\end{document}
