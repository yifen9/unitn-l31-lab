\documentclass[12pt]{article}

\usepackage{graphicx}
\usepackage{amssymb}
\usepackage{amsmath}
\usepackage[margin=1in]{geometry}
\usepackage{fancyhdr}
\usepackage{enumerate}
\usepackage[shortlabels]{enumitem}

\pagestyle{fancy}
\fancyhead[l]{Li Yifeng}
\fancyhead[c]{Homework \#5}
\fancyhead[r]{\today}
\fancyfoot[c]{\thepage}
\renewcommand{\headrulewidth}{0.2pt}
\setlength{\headheight}{15pt}

\newcommand{\bP}{\mathbb{P}}

\begin{document}
	
	\section*{Question 4}
	
	\noindent We are given a bowl that contains $6$ balls, numbered from $1$ to $6$. We extract two balls and denote $X$ and $Y$ the numbers on the ball obtained at the first (second) extraction, and $W = \max(X,Y)$ the maximum value obtained. In all scenarios, describe the probability distribution of $W$.
	
	\bigskip
	
	\begin{enumerate}[start=1,label={\bfseries Part \arabic*:},leftmargin=0in]
		\bigskip\item In the first scenario, assume that the extractions are made with replacement.
		
		\subsection*{Solution}
		
			By applying the principle of symmetry, easy to define the probability space as
			
			\[
			\begin{aligned}
				\Omega &= \{1,2,\dots,6\}\\
				\mathcal{F} &= \mathcal{P}(\Omega)\\
				\bP &:\enspace \bP(\{1\}) = \bP(\{2\}) = \dots = \bP(\{6\}) = \frac{1}{6}
			\end{aligned}
			\]
			
			Then as given, we have
			
			\[
			\begin{aligned}
				X,Y &:\enspace \Omega \rightarrow R\\
				W &= \max(X,Y)\\
				R &= \{1,2,\dots,6\}
			\end{aligned}
			\]
			
			Then we have the probability distribution of $W$
			
			\[
			\begin{aligned}
				p(1) &= \bP(W = 1) &= \bP(\{(1,1)\}) &= \frac{1}{36}\\
				p(2) &= \bP(W = 2) &= \bP(\{(1,2),(2,1),(2,2)\}) &= \frac{3}{36}\\
				p(3) &= \bP(W = 3) &= \bP(\{(1,3),(2,3),\dots,(3,3)\}) &= \frac{5}{36}\\
				p(4) &= \bP(W = 4) &= \bP(\{(1,4),(2,4),\dots,(4,4)\}) &= \frac{7}{36}\\
				p(5) &= \bP(W = 5) &= \bP(\{(1,5),(2,5),\dots,(5,5)\}) &= \frac{9}{36}\\
				p(6) &= \bP(W = 6) &= \bP(\{(1,6),(2,6),\dots,(6,6)\}) &= \frac{11}{36}
			\end{aligned}
			\]
			
			Or a general formula without cases
			
			\[p(w) = \frac{2w-1}{36},\quad w\in \{1,2,\dots,6\}\]
		
		\subsection*{Answer}
		
			\[\boxed{\text{See above.}}\]
		
		\bigskip\item In the second scenario, assume that the extractions are performed without replacement.
		
		\subsection*{Solution}
		
		\subsection*{Answer}
		
			\[\boxed{}\]
			
		\bigskip\item In the third scenario, assume that after the first extraction, we replace the ball in the urn, together with another one with the same number.
		
		\subsection*{Solution}
		
		\subsection*{Answer}
		
			\[\boxed{}\]
	\end{enumerate}
	
\end{document}
