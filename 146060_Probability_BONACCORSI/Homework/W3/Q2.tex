\documentclass[12pt]{article}
\usepackage{amsmath, amssymb}
\usepackage{geometry}
\geometry{a4paper, margin=1in}
\usepackage{setspace}
\usepackage{lmodern}
\usepackage{titlesec}

% Formatting section headings
\titleformat{\section}[block]{\bfseries\Large}{\thesection.}{1em}{}
\titleformat{\subsection}[block]{\bfseries\large}{\thesubsection.}{1em}{}

\title{Question 2: European Roulette Probability Problems}
\author{}
\date{}

\begin{document}
\maketitle
\onehalfspacing

\section*{Problem Statement}
A \emph{European roulette} consists of a wheel with 37 numbered pockets, labeled from $0$ to $36$. The numbers are distributed as described in the following parts.

\bigskip

\section*{Part 1: Probability of an Even Number Pocket}
Let the probability space be defined by
\[
\begin{aligned}
\Omega &= \{0,1,2,\dots,36\},\\[1mm]
\mathcal{F} &= \mathcal{P}(\Omega),\\[1mm]
\mathbb{P} &:\quad \mathbb{P}(\{0\}) = \mathbb{P}(\{1\}) = \cdots = \mathbb{P}(\{36\}) = \frac{1}{37}.
\end{aligned}
\]
Define the set of even-numbered pockets as
\[
S_{\mathrm{even}} = \{2,4,6,\dots,36\}.
\]
Since there are $18$ even numbers in $\Omega$, we have
\[
\mathbb{P}(S_{\mathrm{even}}) = \frac{\# S_{\mathrm{even}}}{\# \Omega} = \frac{18}{37}.
\]

\medskip

\textbf{Answer for Part 1:} 
\[
\boxed{\mathbb{P}(S_{\mathrm{even}}) = \frac{18}{37}}
\]

\bigskip

\section*{Part 2: Probability of a Red Pocket}
Let the set of red pockets be
\[
S_{\mathrm{red}} = \{1,3,5,\dots,35\}.
\]
Assuming there are $18$ red pockets (as given), it follows that
\[
\mathbb{P}(S_{\mathrm{red}}) = \frac{\# S_{\mathrm{red}}}{\# \Omega} = \frac{18}{37}.
\]

\medskip

\textbf{Answer for Part 2:} 
\[
\boxed{\mathbb{P}(S_{\mathrm{red}}) = \frac{18}{37}}
\]

\bigskip

\section*{Part 3: Conditional Probability of an Even Pocket Given a Red Outcome}
We wish to compute the probability
\[
\mathbb{P}_{S_{\mathrm{red}}}(S_{\mathrm{even}}) = \frac{\mathbb{P}(S_{\mathrm{red}} \cap S_{\mathrm{even}})}{\mathbb{P}(S_{\mathrm{red}})}.
\]
Let 
\[
S_{\mathrm{red}} \cap S_{\mathrm{even}} = \{12,14,\dots,36\}.
\]
Then, by counting the number of elements in this intersection relative to the total number of red pockets, we have
\[
\mathbb{P}_{S_{\mathrm{red}}}(S_{\mathrm{even}}) = \frac{\#\{12,14,\dots,36\}}{\# S_{\mathrm{red}}} = \frac{4}{9}.
\]

\medskip

\textbf{Answer for Part 3:} 
\[
\boxed{\mathbb{P}_{S_{\mathrm{red}}}(S_{\mathrm{even}}) = \frac{4}{9}}
\]

\end{document}