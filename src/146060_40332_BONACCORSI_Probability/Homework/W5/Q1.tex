\documentclass[12pt]{article}

\usepackage{graphicx}
\usepackage{amssymb}
\usepackage{amsmath}
\usepackage[margin=1in]{geometry}
\usepackage{fancyhdr}
\usepackage{enumerate}
\usepackage[shortlabels]{enumitem}

\pagestyle{fancy}
\fancyhead[l]{Li Yifeng}
\fancyhead[c]{Homework \#5}
\fancyhead[r]{\today}
\fancyfoot[c]{\thepage}
\renewcommand{\headrulewidth}{0.2pt}
\setlength{\headheight}{15pt}

\newcommand{\bP}{\mathbb{P}}

\begin{document}
	
	\section*{Question 1}
	
	\noindent Consider a binary communication channel, with input $X$ having a Bernoulli distribution with parameter $p = 0.9$. The common error probability is $\epsilon = 0.05$ (i.e., the probability that the received character differs from the input character is $\epsilon$). Let $Y$ denote the output character.
	
	\bigskip
	
	\begin{enumerate}[start=1,label={\bfseries Part \arabic*:},leftmargin=0in]
		\bigskip\item Show that $Y$ is a Bernoulli distribution with parameter $q$.
		
		\subsection*{Solution}
		
			We know that $X$ has a Bernoulli distribution in sample space $\Omega = \{0,1\}$ (let 0 denotes $not\enspace received$ and 1 denotes $received$) with parameter $p = 0.9$, then easy to see that
			
			\[
			\begin{aligned}
				\bP(Y=0) &= \bP(X=0)\bP_{X=0}(Y=0) + \bP(X=1)\bP_{X=1}(Y=0)\\
				&= (1-p)(1-\epsilon) + p\epsilon\\
				&= 2p\epsilon - p - \epsilon + 1
			\end{aligned}
			\]
			
			Similiarly, we have
			
			\[
			\begin{aligned}
				\bP(Y=1) &= \bP(X=0)\bP_{X=0}(Y=1) + \bP(X=1)\bP_{X=1}(Y=1)\\
				&= (1-p)\epsilon + p(1-\epsilon)\\
				&= -2p\epsilon + p + \epsilon
			\end{aligned}
			\]
			
			Then we have
			
			\[
				0 < \bP(Y=0), \bP(Y=1) < 1,\quad\text{and}\quad \bP(Y=0) + \bP(Y=1) = 1
			\]
			
			Then we can say that $Y$ is a Bernouli distribution in sample space $\Omega = \{0,1\}$ with parameter
			
			\[
			\begin{aligned}
				q &= \bP(Y = 1)\\
				&= -2p\epsilon + p + \epsilon
			\end{aligned}
			\]
		
		\subsection*{Answer}
		
			\[\boxed{\text{See above.}}\]
		
		\bigskip\item Determine $q$.
		
		\subsection*{Solution}
		
			With the explanation and formula above, easy to see that
		
			\[q = -2p\epsilon + p + \epsilon = 0.86\]
		
		\subsection*{Answer}
		
			\[\boxed{q = 0.86}\]
			
		\bigskip\item Compute the joint probability distribution function of $(X,Y)$.
		
		\subsection*{Solution}
		
			Easy to compute the joint distribution function of $(X,Y)$ as
			
			\[
				f(x,y) =
					\begin{cases}
						\begin{aligned}
							\bP(X=0,Y=0) &= \bP(X=0)\bP_{X = 0}(Y = 0) = &(1-p)1 - \epsilon &= 0.095,&\quad x = 0,\quad y = 0\\
							\bP(X=0,Y=1) &= \bP(X=0)\bP_{X = 0}(Y = 1) = &(1-p)\epsilon &= 0.005,&\quad x = 0,\quad y = 1\\
							\bP(X=1,Y=0) &= \bP(X=1)\bP_{X = 1}(Y = 0) = &p\epsilon &= 0.045,&\quad x = 1,\quad y = 0\\
							\bP(X=1,Y=1) &= \bP(X=1)\bP_{X = 1}(Y = 1) = &p(1 - \epsilon) &= 0.855,&\quad x = 1,\quad y = 1
						\end{aligned}
					\end{cases}
			\]
		
		\subsection*{Answer}
		
			\[\boxed{f(x,y) =
				\begin{cases}
					\begin{aligned}
						0.095,&\quad x = 0,\quad y = 0\\
						0.005,&\quad x = 0,\quad y = 1\\
						0.045,&\quad x = 1,\quad y = 0\\
						0.855,&\quad x = 1,\quad y = 1
					\end{aligned}
			\end{cases}}\]
	\end{enumerate}
	
\end{document}
