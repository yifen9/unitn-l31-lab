\documentclass[12pt]{article}

\usepackage{graphicx}
\usepackage{amssymb}
\usepackage{amsmath}
\usepackage[margin=1in]{geometry}
\usepackage{fancyhdr}
\usepackage{enumerate}
\usepackage[shortlabels]{enumitem}

\pagestyle{fancy}
\fancyhead[l]{Li Yifeng}
\fancyhead[c]{Homework \#8}
\fancyhead[r]{\today}
\fancyfoot[c]{\thepage}
\renewcommand{\headrulewidth}{0.2pt}
\setlength{\headheight}{15pt}

\newcommand{\bE}{\mathbb{E}}
\newcommand{\bP}{\mathbb{P}}

\begin{document}
	
	\section*{Question 1}
	
	\noindent Most people who are investing their money want to do so at a very limited risk. For this reason many pension funds and mutual funds require that any stock or bond purchased has to be rated to have an extremely low risk of going into default. Standard and Poors (S\&P) and Moody are two companies who rate the risk of investments, and their triple A ranking is supposed to indicate that the bond has a risk of less than $1$ in $850$ ($0.12\%$) of going into default.
	
	\bigskip
	
	\noindent It is rather difficult to find individual investments which are this conservative in their risk. Therefore, it is common practice on Wall Street to bundle together a bunch of riskier investments into what are called Collateralized Debt Obligations (CDOs). The idea is that these bonds pay out as long as not every loan in the bundle fails.
	
	\bigskip
	
	\begin{enumerate}[start=1,label={\bfseries Part \arabic*:},leftmargin=0in]
		\bigskip\item What is the probability of a CDO failing to pay out if it consists of $10$ loans, each with independent failure probability of $0.1$?
		
		\subsection*{Solution}
		
			Since the CDO fails only if \emph{all} loans default and failures are independent,
			
			\[
				\bP(\text{CDO fails}) = (0.1)^{10} = 10^{-10}
			\]
		
		\subsection*{Answer}
		
			\[\boxed{\bP(\text{CDO fails}) = 10^{-10}}\]
		
		\bigskip\item How many independent subprime loans, each with failure probability $0.5$, are needed to create a triple-A CDO (failure risk $< 0.0012$)?
		
		\subsection*{Solution}
		
			We require
			
			\[
				(0.5)^n < 0.0012
			\]
			
			Taking natural logarithms,
			
			\[
				n\ln(0.5) < \ln(0.0012)
				\quad\Longrightarrow\quad
				n > \frac{\ln(0.0012)}{\ln(0.5)} \approx 9.71
			\]
			
			Thus the smallest integer $n$ is $10$.
		
		\subsection*{Answer}
		
			\[\boxed{10}\]
	\end{enumerate}
	
\end{document}
