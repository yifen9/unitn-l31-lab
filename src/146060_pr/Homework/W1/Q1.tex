\documentclass[a4paper,12pt]{article}
\usepackage{amsmath, amssymb, geometry}
\geometry{left=1in, right=1in, top=1in, bottom=1in}

\title{Statistical Analysis of Newborn Male Infants' Birth Weights}
\author{}
\date{}

\begin{document}

\maketitle

\section{Introduction}
A sample of newborn male infants had their birth weight measured in ounces. The data was then grouped into classes, where $Z_k$ represents the class midpoint and $N_k$ represents the frequency of the $k$-th class.

\section{Number of Individuals}
Let the number of classes be:
\begin{equation}
K = 15
\end{equation}
Then the total number of individuals is:
\begin{equation}
n = \sum_{k=1}^{K} N_k = 2+6+\dots+1 = 9,465.
\end{equation}

\subsection*{Answer}
\begin{itemize}
    \item $n = 9,465$
\end{itemize}

\section{Mean and Variance}
Let the individual values be denoted as:
\begin{equation}
x_i, \quad i \in \{1,2,\dots,n\}.
\end{equation}

The relative frequency is given by:
\begin{equation}
p_k = \frac{N_k}{n}.
\end{equation}

The mean is computed as:
\begin{equation}
\bar{x} = \sum_{k=1}^{K} p_k Z_k = \frac{1}{n} \sum_{k=1}^{K} N_k Z_k = \frac{1,035,467}{9,465} \approx 109.400.
\end{equation}

The variance is given by:
\begin{equation}
\sigma^2 = \sum_{k=1}^{K} p_k (Z_k - \bar{x})^2 = \frac{1}{n} \sum_{k=1}^{K} N_k (Z_k - \bar{x})^2 = \frac{1,748,957}{9,465} \approx 184.782.
\end{equation}

\subsection*{Answer}
\begin{itemize}
    \item $\bar{x} = 109.400$
    \item $\sigma^2 = 184.782$
\end{itemize}

\section{Conversion to Grams}
Each individual weight in ounces is converted to grams:
\begin{equation}
y_i = 28.349 \cdot x_i.
\end{equation}

The new mean and variance become:
\begin{align}
\bar{y} &= 28.349 \cdot \bar{x} = \frac{29,354,453,983}{9,465,000} \approx 3,101.369, \\
\sigma_y^2 &= 28.349^2 \cdot \sigma^2 = \frac{140,557,692,831}{946,500} \approx 148,502.581.
\end{align}

\subsection*{Answer}
\begin{itemize}
    \item $\bar{y} = 3,101.369$
    \item $\sigma_y^2 = 148,502.581$
\end{itemize}

\section{Quartiles and Median}
The index of the median is:
\begin{equation}
i_m = \frac{n+1}{2} = 4,733.
\end{equation}

The median class satisfies:
\begin{equation}
\sum_{k=1}^{6} N_k = 3,049 < i_m < 5,289 = \sum_{k=1}^{7} N_k.
\end{equation}

The median is calculated as:
\begin{equation}
m = Z_6 + (Z_7 - Z_6) \left( \frac{i_m - \sum_{k=1}^{6} N_k}{N_7} \right) \approx 105.014.
\end{equation}

Similarly, the first and third quartiles are computed using:
\begin{align}
i_{Q_1} &= \frac{n}{4} + \frac{1}{2} = 2,366.75, \\
i_{Q_3} &= \frac{3}{4} n + \frac{1}{2} = 7,099.25.
\end{align}

The quartile classes satisfy:
\begin{align}
\sum_{k=1}^{5} N_k = 1,320 &< i_{Q_1} < 3,049 = \sum_{k=1}^{6} N_k, \\
\sum_{k=1}^{7} N_k = 5,289 &< i_{Q_3} < 7,296 = \sum_{k=1}^{8} N_k.
\end{align}

The first and third quartiles are given by:
\begin{align}
Q_1 &= Z_5 + (Z_6 - Z_5) \left( \frac{i_{Q_1} - \sum_{k=1}^{5} N_k}{N_6} \right) \approx 96.449, \\
Q_3 &= Z_7 + (Z_8 - Z_7) \left( \frac{i_{Q_3} - \sum_{k=1}^{7} N_k}{N_8} \right) \approx 114.216.
\end{align}

\subsection*{Answer}
\begin{itemize}
    \item $m_p = 105.014$
    \item $Q_1 \approx 96.449$
    \item $Q_2 = 105.014$
    \item $Q_3 \approx 114.216$
\end{itemize}

\end{document}
