\documentclass{article}
\usepackage{amsmath}
\usepackage{enumitem}

\title{Abstract Machines}
\author{}
\date{}

\begin{document}

\maketitle

\section*{1. Three Examples of Abstract Machines in Different Contexts}

\begin{enumerate}
    \item \textbf{Java Virtual Machine (JVM):} An abstract machine designed to execute Java bytecode. It provides a platform-independent execution environment and includes components such as a class loader, memory management, and a runtime execution engine.
    \item \textbf{SECD Machine:} A theoretical abstract machine designed for evaluating expressions in the lambda calculus, widely used in functional language implementations such as Lisp.
    \item \textbf{x86 Instruction Set Architecture (ISA) Emulator:} An abstract machine used to simulate x86 CPU instructions, allowing execution of x86 binaries on non-x86 hardware, such as QEMU.
\end{enumerate}

\section*{2. Functioning of an Interpreter for a Generic Abstract Machine}

An interpreter for an abstract machine follows these steps:

\begin{enumerate}[label=\textbf{\arabic*.}]
    \item \textbf{Fetch:} Retrieve the next instruction from memory.
    \item \textbf{Decode:} Analyze the instruction and determine the operation.
    \item \textbf{Execute:} Perform the operation by updating registers, memory, or stack.
    \item \textbf{Repeat:} Continue execution until a halt instruction or an error occurs.
\end{enumerate}

For example, in a stack-based machine, an interpreter fetches bytecode instructions, manipulates the stack accordingly, and processes function calls and control flow.

\section*{3. Differences Between Interpretative and Compiled Implementations}

\begin{table}[h!]
\centering
\begin{tabular}{|p{2cm}|p{4cm}|p{4cm}|}
\hline
\textbf{Feature} & \textbf{Interpretation} & \textbf{Compilation} \\
\hline
Execution Method & Runs source code directly using an interpreter. & Translates the entire source code into machine code before execution. \\
\hline
Performance & Slower due to real-time code evaluation. & Faster as execution happens on precompiled machine code. \\
\hline
Portability & More portable since the same interpreter can run on multiple platforms. & Less portable as compiled code is specific to the target architecture. \\
\hline
Debugging & Easier, as execution happens step by step. & Harder, as debugging requires additional tools like debuggers. \\
\hline
Example Languages & Python, JavaScript, Ruby & C, C++, Rust \\
\hline
\end{tabular}
\end{table}

\textbf{Advantages of Interpretation:}
\begin{itemize}
    \item Easier to debug.
    \item More flexible, enabling dynamic execution.
\end{itemize}

\textbf{Advantages of Compilation:}
\begin{itemize}
    \item Faster execution.
    \item Optimized for specific hardware.
\end{itemize}

\section*{4. Using an Abstract Machine \( C \) to Implement Another Abstract Machine for Language \( L \)}

\begin{enumerate}
    \item Implement an interpreter for \( L \) that runs on \( C \).
    \item Write a compiler that translates \( L \) programs into the instruction set of \( C \).
    \item Execute compiled programs using \( C \).
\end{enumerate}

This approach allows leveraging \( C \)’s infrastructure (memory management, execution model) to implement \( L \) efficiently.

\section*{5. Advantages of Using an Intermediate Machine for the Implementation of a Language}

\begin{itemize}
    \item \textbf{Portability:} Programs can be compiled into the intermediate machine’s code and run on different hardware.
    \item \textbf{Optimization:} The compiler can perform optimizations at the intermediate code level.
    \item \textbf{Simplified Compilation:} Instead of writing compilers for multiple architectures, one backend for the intermediate machine suffices.
    \item \textbf{Runtime Analysis:} Intermediate representations facilitate Just-In-Time (JIT) optimizations and debugging.
\end{itemize}

Examples include the JVM for Java and LLVM IR for multi-language compilation.

\section*{6. Obtaining a Compiled Implementation from an Interpretative One}

To transform an interpretative Pascal implementation into a compiled one:

\begin{enumerate}
    \item Modify the Pascal compiler to output native machine code instead of P-code.
    \item Use the Pascal compiler (which produces P-code) to compile the modified compiler into P-code.
    \item Interpret this modified compiler using the P-code interpreter, producing a native compiler.
    \item Use this new native compiler to compile Pascal programs directly into machine code.
\end{enumerate}

This technique, known as \textit{bootstrapping}, minimizes manual effort while transitioning from an interpretative to a compiled implementation.

\section*{7. Futamura’s First Projection}

Given an interpreter \( IL_{L1}(X, Y) \) that interprets a program \( X \) written in \( L_1 \) on language \( L \):

\begin{enumerate}
    \item Partial evaluation means specializing \( IL_{L1} \) with respect to program \( P \), effectively transforming the interpreter into a compiler.
    \item The result, \( Peval_L(IL_{L1}, P) \), is a program that directly executes the logic of \( P \) without requiring interpretation. This is equivalent to compiling \( P \) into an efficient executable for \( L \).
\end{enumerate}

This idea underlies Just-In-Time (JIT) compilation and compiler generation techniques.

\end{document}
